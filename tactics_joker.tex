% Tactics Joker.

\documentclass{article}

\title{Tactics Joker}
%\author{Jon Sagotsky}

\setlength{\parindent}{0.0in}
\setlength{\parskip}{0.1in}

\begin{document}
%\maketitle
%%%%%%%%%%%%%%%%%%%%%%%%%%%%%%%%%%%%%%%%%%%%%%%%%%%%%%%%%%%%%%%%%%%%
%  V A R I A B L E S
%
% This game is still being tweaked.  Most of the values were pulled
% from the ether.  They seem sane so far.  Let's not assume they 
% remain sane just for the sake of convenience.
%
\def \maxarmy {6 }            % most units in a single army
\def \reinforcements {15 }    % most units per player, on and off the board
\def \startarmy {5 }          % starting army to place at a castle
\def \recruitroll {6 }        % roll this or higher to recruit each turn
%%%%%%%%%%%%%%%%%%%%%%%%%%%%%%%%%%%%%%%%%%%%%%%%%%%%%%%%%%%%%%%


\huge
Tactics Joker

\normalsize

\section{Setup}


Ingredients: 
\begin{itemize}
  \item 1 deck of playing cards.  Jokers included.
  \item 6 six sided dice.  You can play with fewer, but more will be convenient.
  \item 2 groups of \reinforcements tokens.  It doesn't matter what they are, but you should be able to fit several on a playing card.  Pennies vs dimes, paper clips vs staples, etc.
  \item 2 players.  More may be possible, but the logistics of that are up to you.
\end{itemize}

Shuffle the deck of cards and lay it out in a 7x7 grid.  Put a king at each of the corners.  These are your castles - the game is won by capturing three of these.

Player 1 selects a castle and puts \startarmy tokens on it.  Player 2 selects 2 castles and puts \startarmy on each of those.  Player 1 gets to put \startarmy tokens on the final castle.  The remaining tokens are your reserves.  They may get played later, but no more than \reinforcements tokens can ever be used by either player.

\section{Turns}

\subsection{Recruitment}

At the beginning of each turn you can try to recruit more tokens from your reinforcements.  Each castle you control can roll a die.  On a result of \recruitroll you can add a token from your reinforcements to the castle.   Note that to control a castle you need to have tokens on it.

\subsection{Movement}

During your turn you can choose three groups of tokens to move across the board.  Pick a card and move tokens from it to an adjacent card.  Diagonal movement is allowed.  You don't have to move all the tokens when you pick a card - tokens can be left behind.  You can spend your movement on the same group of tokens twice in one round, but not three times.  

Black cards, clubs and spades, represent forested and mountainous terrain respectively.  Because they are harder to move through they are considered to be rough terrain.  This means that when you move tokens into one of cards, it's more difficult.  Each token moving onto the rough terrain has to roll a six sided die.  On a high roll (4-6) they enter the terrain.  Otherwise they're left behind.  Mountains and forests have benefits which are explained in the Terrain section. 

Whenever your tokens land on a square with enemy units, combat happens.

\section{Combat}

\subsection{Basics}

Combat is straightforward.  Each side rolls a d6 for each token on their side of the fight.  Add these together.

Next, add each army's terrain bonus:  The defending army adds the value of their card the fight is taking place on and the attackers use the value of the card they attacked from.  Aces are valued at 1 point.  Jacks and queens are 10.  Kings are the best defended cards on the board and are worth 13.

Finally, add support for nearby allies.  Each token that's on a card adjacent to the combat adds a point to his side's army.  

Whomever rolled higher wins the fight.  The loser automatically loses one of his tokens, plus one per every five points by which he was defeated.  If the attacker lost, he is pushed back to the square he attacked from.  If the defender lost, he is pushed directly back. 

In the event of a tie, losses are heavy.  Both sides lose a token and then reroll.

\subsection{Notes}

Attacking into rough terrain requires a roll as usual.  This means that on average, only half your tokens will make it into the fight - tokens left behind don't get to roll!

Supporting tokens only add 1 point to combat, while those in the combat can add 1-6 points.  However supporting tokens won't die if you roll badly.

Tokens that are killed return to your reinforcement pool.  You can only ever have \reinforcements tokens in play at a time, but if they die they can be replaced.

Just like movement, supporters are considered adjacent if they're next to or diagonal to a card.  

\section{Terrain}

Different types of terrain affect how combat works.  In addition to being rough terrain, forests and mountains have the following effects:

\subsection{Plains - Diamonds and Hearts}

Nothing special.

\subsection{Forest - Clubs}

Forested terrain is easily defended.  When someone attacks into a forest from a non-forest, they get no bonus from terrain.  

\subsection{Mountain - Spades}

Mountains provide high ground which is advantageous for attackers.  Attacking a non-mountain from a mountain removes the defender's terrain bonus.

Note that attacking a forest from a mountain results in both penalties - nobody gets a terrain bonus! 

\subsection{Conscripts - Jacks and Queens}

Non-king face cards provide additional units.  When you land on a jack or queen, add a unit to that army.  Then flip over the card, as it's depleted.

Jacks and Queens, flipped or not, do not count as forests or mountains, despite their suit.  They always provide a 10 point terrain bonus.  

\subsection{Castles - Kings}

Kings also do not count as a forest or mountain, but provide a 13 point terrain bonus.  Castles are the starting point for the game.  They are also the ending point.  When a player starts his turn controlling three castles, he wins.

\subsection{Jokers - Wild}

Jokers are wild cards.  If you land on a joker, look at the 5 cards that were not included in the grid.  Pick one of those and replace the joker with it.

\section{Winning}

The game is won when a player begins his turn controlling 3 of the 4 castles.  Control is defined by occupying that castle with at least one unit.


\newpage
\section{Optional Rules}

These rules are entirely unplaytested.  Consider them brainstorming rather than rules.

\subsection{Races}

Players can randomly choose one of the following races.  Each race gets a special ability to make them distinct.

\subsubsection{Elves}

Elves can enter forest terrain on a roll of 2-6.

\subsubsection{Dwarves}

Dwarves can enter mountain terrain on a roll of 2-6.

\subsubsection{Kobolds}

When rolling to add new units, kobolds are added on 4-6 instead of just 6.  Kobolds have an extra three tokens to recruit.

Kobolds are weaker fighters than the other races.  During combat checks, disregard 6s.

\subsubsection{Humans}

Humans are the standard race and should have no bonus.  However since all the other races are bonus based...

Humans have an extra point of movement which can only be used to move a one unit army at a time.  

\subsubsection{Undead}

When you kill a unit, roll a die.  On a high roll, add that unit to your army.  

One less point of movement per turn.

\subsection{Fog of war}

All cards are face down.  When a card becomes occupied, flip it over.   At any time you can look at a card adjacent to a card you occupy.  On a mountain top, you can look two cards away, due to your higher vantage point.

\subsection{D\&D 4e Skill Challenge}

If one were to use tactics joker as a substitute for mass combat in Dungeons and Dragons 4th edition, it could be suplemented with the following skill checks.  A skill check can be made in place of a move action once per turn.  Skills that target an opponent use the hard difficult.  Skills that target terrain use normal difficulty plus the value of the terrain.

You can use these rules in one of two ways.  Each PC can be a unit in the army or a PC can control the army.

If a PC is in command, he can use the abilities anywhere appropriate.  Otherwise play as normal.

If each PC acts as a unit, put them on the board with the other tokens.  PCs don't roll in combat, instead they add their character level.  (Don't forget to include NPCs of similar levels!)  Furthermore, a PC using a skill can only use that skill in his square or an adjacent one.

\subsubsection{Acrobatics}

Moving into a forest does not negate your terrain bonus when attacking.

\subsubsection{Arcana}

Swap the position of two cards adjacent to one of your armies.  The difficulty of this check is dependent on the greater of the two cards.  

\subsubsection{Athletics}

The terrain you're defending does not lose its terrain bonus to an attack from mountains next turn.

\subsubsection{Bluff}

Until your next turn, if you successfully defend against an enemy attack you may move that enemy army into a space adjacent to the one they were in when they started their turn.  That's where they thought you were all along!

\subsubsection{Diplomacy}

Choose an army.  They may not be attacked until the start of your next turn.

\subsubsection{Dungeoneering}

You find a passage to the Underdark in the space you're entering.  The Underdark is rough terrain and provides no terrain bonus, reinforcements, or castles, but so long as you're in the Underdark, you can't fight armies above ground and can exist in the same space as them.  Exiting the Underdark also requires a dungeoneering roll.

\subsubsection{Heal}

After losing a unit, roll a 5 or 6.  That unit does not die.

\subsubsection{Insight}

Pick one of your armies.  Roll a combat check.  Don't show it to your opponent.  Until your next turn, that is what that army rolls for its next combat.

\subsubsection{Intimidate}

Pick one of your armies.  Until the start of your next turn, the terrain they occupy and all adjacent cards are considered rough terrain to your enemies.

\subsubsection{Nature}

For the rest of the turn, ignore rough terrain.

\subsubsection{Perception}

Pick an enemy army.  They roll for combat.  If they engage in combat this turn, they must use that roll.

\subsubsection{Religion}

On your next attack this turn, reroll a die.

\subsubsection{Stealth}

Swap the position of two of your armies.  The difficulty is Hard plus the number of cards between the two armies.  Turns out they'd been misleading the enemy all along.

\subsubsection{Streetwise}

Roll a 5 or 6.  Gain a second unit from rolling reinforcements or picking up a face card.

\subsubsection{Theivery}

The next time your opponent attacks, you may reroll one of his dice.

\newpage
\section{Author's Notes}

This is a work in progress.  Here are some notes.

Fog of war sucks.  Cards are annoying to flip.  Without knowing where you're going, it's hard to be tactical early on.  Originally this was a base rule.  It's optional because I still like mountain based perception bonuses, but it is not fun to use.

Originally there was no diagonal movement.  There weren't enough movement options.  Also, mountains and forests were way too good, since it was harder to go around them.

I tried keeping the castles closer to the center or randomizing them.  That made the game go really quickly.  It also put the resources (face cards) closer to the starting point, so those burned more quickly.

There is the potential to sit around trading castles.  Not sure what to do about that.

Biggest problem so far is the win condition.  Take all the castles was too slow.  Eradicate the enemeis eneded in a cat and mouse chase.  Take 3 castles seems decent, but more playtesting would be nice.  Occupying bases seems necessary to motivate players to stay on them.

Originally recruiting could be done in place of a move action.  The result was the loser spent all his actions recruiting.  Maybe this could be limited?  Face cards as recruits came from this, as that limits how many are available, but pushes the armies out the door in interesting directions.  Debating respawning/flipping face cards once all are spent.

Still unsure if face cards should be 10 points.  Aside from castles and the black 10s, that makes them the best cards.  Is getting an extra unit enough that they could scale back?

Forests and mountains are crazy powerful.  Might cut the penalty in half instead of entirely negating terrain.

Support works weirdly with rough terrain.  
\end{document}
