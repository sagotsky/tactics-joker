% Tactics Joker.

\documentclass{article}

\title{Tactics Joker}
\author{Jon Sagotsky}

\setlength{\parindent}{0.0in}
\setlength{\parskip}{0.1in}

\begin{document}
\maketitle
%%%%%%%%%%%%%%%%%%%%%%%%%%%%%%%%%%%%%%%%%%%%%%%%%%%%%%%%%%%%%%%%%%%%
%  V A R I A B L E S
%
% This game is still being tweaked.  Most of the values were pulled
% from the ether.  They seem sane so far.  Let's not assume they 
% remain sane just for the sake of convenience.
%
\def \maxarmy {6 }            % most units in a single army
\def \reinforcements {15 }    % most units per player, on and off the board
\def \recruitroll {6 }        % roll this or higher to recruit each turn
%%%%%%%%%%%%%%%%%%%%%%%%%%%%%%%%%%%%%%%%%%%%%%%%%%%%%%%%%%%%%%%


Tactics Joker

This is an attempt at a tactical board game using commodity gaming pieces.  It is intended for two players, but more may work.  This game is highly inspired by A Game of Thrones.  It is loosely inspired by Dragon Dice, which I've never played but still have some ideas about.  There might also be a bit of Small Worlds thrown in to the optional rules.


\section{Setup}

Ingredients: 
\begin{itemize}
  \item 1 deck of playing cards.  Jokers included.
  \item 6 six sided dice.  You can play with fewer, but more will be convenient.
  \item 2 groups of \reinforcements tokens.  It doesn't matter what they are, but you should be able to fit several on a playing card.  Pennies vs dimes, paper clips vs staples, etc.
  \item 2 players.  More may be possible, but the logistics of that are up to you.
\end{itemize}

Shuffle the deck of cards and lay it out in a 7x7 grid, with a king card at each of the corners.  The cards represent terrain on a battlefield.  Red cards, diamonds and hearts, are flat plains.  Spades are forests and clubs are mountains.  Kings are castles.  Queens and jacks are new recruits.  Jokers are of course wild and will be covered later.  Aces have a value of 1.  Jacks and queens have a value of 10, but kings are 13.

Each side has a set of \reinforcements tokens representing their combat units.  A group of dice is called an army.  An army can never be bigger than \maxarmy units, so spread out.  Each turn a player picks and moves several of his armies.  If an enemy unit occupies that piece of terrain, they fight.

\section{Winning}

The game is won when a player begins his turn controlling 3 of the 4 castles.  Control is defined by occupying that castle with at least one unit.

\section{Combat}

When opposing armies enter the same space, combat happens.  Both sides roll dice and figure out what their score is.  

\begin{itemize}
  \item Each player rolls a 6 sided die for each of the units in his army.  
  \item Add the terrain bonus.  The defender uses the card the battle is taking place on.  The attacker uses the terrain he used to launch his attack.
  \item Add 1 for each supporting unit.  A unit provides support if it's in a space adjacent to the battle. 
\end{itemize}

Whomever scores higher wins.  One of the loser's units dies, plus one more for every 5 points difference between the combat scores.  If the defender lost, his remaining army is pushed backward (if this isn't an option, the attacker chooses where the defender is pushed).  If the attacker lost, his units return to the space they were in before the attack.

In the event of a tie, losses are heavy.  Both sides lose a unit and reroll.

\section{Movement}

A player gets three moves a turn.  You can move an army, or a portion of an army to an adjacent card.  Diagonal movement is okay.  No one unit may move more than two spaces in a single turn.

The different cards represent different types of terrain and have different tactical advantages.

Plains are plain.  Nothing special here.  \emph{Hearts and diamonds.}

Forests are difficult to move through but easily defended.  Entering a forest from a non-forest space requires each unit entering the space to roll high on a die roll.  Armies hiding in the forest are sheltered from their attackers - the attacker gains no terrain bonus, unless he is attacking from another forest space. \emph{Clubs.}

Mountains are also difficult to move through, requiring a high die roll to enter from a non mountain space.  They provide high ground.  When an attack is initiated from a mountain, the defender gets no terrain bonus to defense unless he is defending a mountain.  \emph{Spades.}

Face cards are additional units that may be picked up.  Face cards do not have a suit, so no rolls are needed to enter those spaces.  When you land on a queen or jack card, add a unit to that army.  Then flip the face card over.  Recruits can only be collected once before they are depleted.  Face cards are always worth 10 points as combat squares.  \emph{Queens and jacks.}

Note that failing to move through difficult terrain still counts as a move action and works against the unit's 2 moves per turn limit.  Being recruited also counts against that limit.

Jokers are wild cards.  When you land on a joker, choose a card from the leftover stack of cards, and replace the joker with that.  \emph{Jokers.}


\section{Recruitment}

In addition to terrain based recruitment, each castle has the opportunity to produce troops.  At the beginning of each turn, if you have enough units left in your reinforcements, roll a die for each castle you control.  On a \recruitroll, add an additional unit to that castle.

\section{Summary - Playing the Game}

Set up the cards as described in setup.  7x7 grid, with a king in each corner.

Now players choose their starting positions.  Player one picks one of the castles.  Then player two gets the next two castles.  Player one gets the last castle, but gets teh first turn.

Each turn, roll for reinforcments at each castle.

Each player gets three movement actions.  No unit may move more than twice in a turn.  If you enter the same space as an opposing army, fight.

When a player begins the turn with units on three castles, that player wins.  

\newpage
\section{Optional Rules}

These rules are entirely unplaytested.  Consider them brainstorming rather than rules.

\subsection{Races}

Players can randomly choose one of the following races.  Each race gets a special ability to make them distinct.

\subsubsection{Elves}

Elves can enter forest terrain on a roll of 2-6.

\subsubsection{Dwarves}

Dwarves can enter mountain terrain on a roll of 2-6.

\subsubsection{Kobolds}

When rolling to add new units, kobolds are added on 4-6 instead of just 6.  Kobolds have an extra three tokens to recruit.

Kobolds are weaker fighters than the other races.  During combat checks, disregard 6s.

\subsubsection{Humans}

Humans are the standard race and should have no bonus.  However since all the other races are bonus based...

Humans have an extra point of movement which can only be used to move a single unit at a time.  

\subsubsection{Undead}

When you kill a unit, roll a die.  On a high roll, add that unit to your army.  

One less point of movement per turn.

\subsection{Fog of war}

All cards are face down.  When a card becomes occupied, flip it over.   At any time you can look at a card adjacent to a card you occupy.  On a mountain top, you can look two cards away, due to your higher vantage point.

\newpage
\section{Author's Notes}

This is a work in progress.  Here are some notes.

Fog of war sucks.  Cards are annoying to flip.  Without knowing where you're going, it's hard to be tactical early on.  Originally this was a base rule.  It's optional because I still like mountain based perception bonuses, but it is not fun to use.

Originally there was no diagonal movement.  There weren't enough movement options.  Also, mountains and forests were way too good, since it was harder to go around them.

I tried keeping the castles closer to the center or randomizing them.  That made the game go really quickly.  It also put the resources (face cards) closer to the starting point, so those burned more quickly.

There is the potential to sit around trading castles.  Not sure what to do about that.

Biggest problem so far is the win condition.  Take all the castles was too slow.  Eradicate the enemeis eneded in a cat and mouse chase.  Take 3 castles seems decent, but more playtesting would be nice.  Occupying bases seems necessary to motivate players to stay on them.

Originally recruiting could be done in place of a move action.  The result was the loser spent all his actions recruiting.  Maybe this could be limited?  Face cards as recruits came from this, as that limits how many are available, but pushes the armies out the door in interesting directions.  Debating respawning/flipping face cards once all are spent.

Still unsure if face cards should be 10 points.  Aside from castles and the black 10s, that makes them the best cards.  Is getting an extra unit enough that they could scale back?

Forests and mountains are crazy powerful.  Might cut the penalty in half instead of entirely negating terrain.

Support works weirdly with rough terrain.  
\end{document}
